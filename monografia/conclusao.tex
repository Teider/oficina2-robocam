\chapter{Conclusão}

No atual momento do projeto, a comunicação entre \textit{CMUcam3} e \textit{Arduíno} ainda não foi implementada, por esse motivo ainda não foi possível fazer os testes de exploração e identificação do objeto.

Analisando a estrutura do robô desde a parte mecânica, eletrônica e as estratégias para exploração e localização do objeto encontrado, espera-se que o robô consiga encontrar um objeto especifico e de destaque dentro de um ambiente pequeno, com as dimensões 168,2 cm por 118,9 cm.

Espera-se também conseguir implementar corretamente um código de exploração inteligente baseado em representações locais para locomoção do robô dentro do espaço determinado. Além disso, conseguir aplicar um código para reconhecer o objeto a ser encontrado utilizando características de destaque (cores fortes, forma, etc.) do objeto. Tais características do robô (locomoção e identificação) serão controladas pela câmera acoplada ao robô, a \textit{CMUcam3}.

Como continuação deste projeto, seria interessante ver o robô se locomovendo em espaços maiores e mais dinâmicos e também que o robô possa identificar objetos mais familiares ao dia a dia como algum equipamento ou molho de chaves.
