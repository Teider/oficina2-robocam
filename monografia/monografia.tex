%% Exemplo de utilizacao do estilo de formatacao normas-utf-tex (http://normas-utf-tex.sourceforge.net)
%% Autores: Hugo Vieira Neto (hvieir@utfpr.edu.br)
%%          Diogo Rosa Kuiaski (diogo.kuiaski@gmail.com)
%% Colaboradores:
%%          Cézar M. Vargas Benitez <cesarvargasb@gmail.com>
%%          Marcos Talau <talau@users.sourceforge.net>


%\documentclass[openright]{normas-utf-tex} %openright = o capitulo comeca sempre em paginas impares
\documentclass[oneside]{normas-utf-tex} %oneside = para dissertacoes com numero de paginas menor que 100 (apenas frente da folha) 

\usepackage[alf,abnt-emphasize=bf,bibjustif,recuo=0cm, abnt-etal-cite=2, abnt-etal-list=99]{abntcite} %configuracao correta das referencias bibliograficas.

\usepackage[brazil]{babel} % pacote portugues brasileiro
\usepackage[utf8]{inputenc} % pacote para acentuacao direta
\usepackage{amsmath,amsfonts,amssymb} % pacote matematico
\usepackage{graphicx} % pacote grafico
\usepackage{times} % fonte times

%Podem utilizar GEOMETRY{...} para realizar pequenos ajustes das margens. Onde, left=esquerda, right=direita, top=superior, bottom=inferior. P.ex.:
%\geometry{left=3.0cm,right=1.5cm,top=4cm,bottom=1cm} 

% ---------- Preambulo ----------
\instituicao{Universidade Tecnol\'ogica Federal do Paran\'a} % nome da instituicao
\programa{Departamento Acadêmico de Eletrônica} % nome do programa
\area{Inform\'atica Industrial} % [Engenharia Biom\'edica] ou [Inform\'atica Industrial] ou [Telem\'atica]

\documento{Monografia} % [Disserta\c{c}\~ao] ou [Tese]
\nivel{Mestrado} % [Mestrado] ou [Doutorado]
\titulacao{Mestre} % [Mestre] ou [Doutor]

\titulo{\MakeUppercase{Robô Explorador de Ambientes}} % titulo do trabalho em portugues
\title{\MakeUppercase{Ambience Explorer Robot}} % titulo do trabalho em ingles

\autor{Luis Guilherme Machado Camargo} % autor do trabalho
\autordois{Marcelo Teider Lopes}
\autortres{Matheus Silva Araújo}
\cita{CAMARGO, Luis Guilherme M. ; LOPES, Marcelo Teider; ARAÚJO, Matheus Silva} % sobrenome (maiusculas), nome do autor do trabalho

\palavraschave{Palavra-chave 1, Palavra-chave 2, ...} % palavras-chave do trabalho
\keywords{Keyword 1, Keyword 2, ...} % palavras-chave do trabalho em ingles

%\comentario{\UTFPRdocumentodata\ apresentada ao \UTFPRdocumentodata\ da \ABNTinstituicaodata\ como requisito parcial para obten\c{c}\~ao do grau de ``\UTFPRtitulacaodata\ em Ci\^encias'' -- \'Area de Concentra\c{c}\~ao: \UTFPRareadata.}

\comentario{\UTFPRdocumentodata\ apresentada ao Departamento Acadêmico de Eletrônica da \ABNTinstituicaodata\ como requisito parcial para aprovação na Disciplina de Oficina de Integração 2.}


\orientador[Orientadora:]{Profa. Dra. Myriam Regattieri De Biase da Silva Delgado} % nome do orientador do trabalho
%\orientador[Orientadora:]{Nome da Orientadora} % <- no caso de orientadora, usar esta sintaxe
%\coorientador{Nome do Co-orientador} % nome do co-orientador do trabalho, caso exista
%\coorientador[Co-orientadora:]{Nome da Co-orientadora} % <- no caso de co-orientadora, usar esta sintaxe
%\coorientador[Co-orientadores:]{Nome do Co-orientador} % no caso de 2 co-orientadores, usar esta sintaxe
%\coorientadorb{Nome do Co-orientador 2}	% este comando inclui o nome do 2o co-orientador

\local{Curitiba} % cidade
\data{\the\year} % ano automatico


%---------- Inicio do Documento ----------
\begin{document}

\capa % geracao automatica da capa
\folhaderosto % geracao automatica da folha de rosto
%\termodeaprovacao % <- ainda a ser implementado corretamente

% dedicatória (opcional)
%\begin{dedicatoria}
%Texto da dedicat\'oria.
%\end{dedicatoria}

% agradecimentos (opcional)
\begin{agradecimentos}
Este trabalhado não teria sido possível sem o projeto anteriormente apresentado por Bruno Meneguele, Fernando Padilha e Vinicius Arcanjo.
Por emprestar o robô e pelos diversos esclarecimentos (muitas vezes sobre assuntos que não os envolviam) nosso muito obrigado.

À Professora Myriam nossos agradecimentos por aceitar o desafio de nos orientar. 

Aos Professores Hugo Vieira e Mário Sérgio.
 
\end{agradecimentos}

% epigrafe (opcional)
%\begin{epigrafe}
%Texto da ep\'igrafe.
%\end{epigrafe}

%resumo
\begin{resumo}
Texto do resumo (m\'aximo de 500 palavras).
\end{resumo}

%abstract
\begin{abstract}
Abstract text (maximum of 500 words).
\end{abstract}

% listas (opcionais, mas recomenda-se a partir de 5 elementos)
\listadefiguras % geracao automatica da lista de figuras
\listadetabelas % geracao automatica da lista de tabelas
\listadesiglas % geracao automatica da lista de siglas
\listadesimbolos % geracao automatica da lista de simbolos

% sumario
\sumario % geracao automatica do sumario


%---------- Inicio do Texto ----------
% recomenda-se a escrita de cada capitulo em um arquivo texto separado (exemplo: intro.tex, fund.tex, exper.tex, concl.tex, etc.) e a posterior inclusao dos mesmos no mestre do documento utilizando o comando \input{}, da seguinte forma:
%\input{intro.tex}
%\input{fund.tex}
%\input{exper.tex}
%\input{concl.tex}


%---------- Primeiro Capitulo ----------
\chapter{Introdução}

\section{Motivação}

\section{Objetivo}

\subsection{Objetivo Geral}

\subsection{Objetivos Específicos}

\section{Metodologia}

%---------- Segundo Capitulo ----------
\chapter{Fundamentos teóricos}

\section{Plataforma Arduíno}

\section{Bússola}

\section{CMUCam3}

\section{Processamento de Imagem}

\section{Mapa Cognitivo}

\section{Tomada de Decisão}


%---------- Terceiro Capitulo ----------
\chapter{Desenvolvimento do Projeto}

\section{Projeto Mecânico}

\section{Integração com a bússola}

\section{Interface Arduíno-CMUCam}

\section{Algoritmo de Decisão}

%---------- Quarto Capitulo ----------
\chapter{Problemas Encontrados}

%---------- Quinto Capitulo ----------
\chapter{Conclusão}

%---------- Referencias ----------
\bibliography{reflatex} % geracao automatica das referencias a partir do arquivo reflatex.bib


%---------- Apendices (opcionais) ----------
\apendice

\chapter{Caderno de Bordo}

%\chapter{Nome do Ap\^endice}

%Use o comando {\ttfamily \textbackslash apendice} e depois comandos {\ttfamily \textbackslash chapter\{\}}
%para gerar t\'itulos de ap\^en-dices.


% ---------- Anexos (opcionais) ----------
%\anexo
%\chapter{Nome do Anexo}

%Use o comando {\ttfamily \textbackslash anexo} e depois comandos {\ttfamily \textbackslash chapter\{\}}
%para gerar t\'itulos de anexos.


% --------- Lista de siglas --------
%\textbf{* Observa\c{c}\~oes:} a lista de siglas nao realiza a ordenacao das siglas em ordem alfabetica
% Em breve isso sera implementado, enquanto isso:
%\textbf{Sugest\~ao:} crie outro arquivo .tex para siglas e utilize o comando \sigla{sigla}{descri\c{c}\~ao}.
%Para incluir este arquivo no final do arquivo, utilize o comando \input{arquivo.tex}.
%Assim, Todas as siglas serao geradas na ultima pagina. Entao, devera excluir a ultima pagina da versao final do arquivo
% PDF do seu documento.


%-------- Citacoes ---------
% - Utilize o comando \citeonline{...} para citacoes com o seguinte formato: Autor et al. (2011).
% Este tipo de formato eh utilizado no comeco do paragrafo. P.ex.: \citeonline{autor2011}

% - Utilize o comando \cite{...} para citacoeses no meio ou final do paragrafo. P.ex.: \cite{autor2011}



%-------- Titulos com nomes cientificos (titulo, capitulos e secoes) ----------
% Regra para escrita de nomes cientificos:
% Os nomes devem ser escritos em italico, 
%a primeira letra do primeiro nome deve ser em maiusculo e o restante em minusculo (inclusive a primeira letra do segundo nome).
% VEJA os exemplos abaixo.
% 
% 1) voce nao quer que a secao fique com uppercase (caixa alta) automaticamente:
%\section[nouppercase]{\MakeUppercase{Estudo dos efeitos da radiacao ultravioleta C e TFD em celulas de} {\textit{Saccharomyces boulardii}}
%
% 2) por padrao os cases (maiusculas/minuscula) sao ajustados automaticamente, voce nao precisa usar makeuppercase e afins.
% \section{Introducao} % a introducao sera posta no texto como INTRODUCAO, automaticamente, como a norma indica.


\end{document}
